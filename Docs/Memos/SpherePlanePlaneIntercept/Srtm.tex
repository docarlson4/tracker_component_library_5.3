\documentclass{article}
\usepackage{amsmath}
\usepackage{graphicx}
\usepackage{hyperref}

\title{\bf{Estimating Height Uncertainty in SRTM Data}}
\author{}
\date{}

\begin{document}

\maketitle

\section{Introduction}
This document discusses how to estimate height uncertainty in Shuttle Radar Topography Mission (SRTM) data for a region defined by latitude and longitude. 

\section{Obtaining SRTM Data}
\begin{itemize}
    \item Download SRTM data in GeoTIFF format, typically with 1-arcsecond (~30m) or 3-arcsecond (~90m) resolution, from sources like \href{https://earthexplorer.usgs.gov/}{USGS Earth Explorer} or \href{https://doi.org/10.5067/MEaSUREs/SRTM/SRTMGL1.003}{NASA’s SRTM data archive}.
    \item Load the data into MATLAB using the \texttt{readgeoraster} function:
\end{itemize}

\begin{verbatim}
[Z, R] = readgeoraster('srtm_file.tif');
\end{verbatim}

\section{Extracting Elevation Data for a Region}
Define your region of interest by specifying latitude and longitude bounds. Extract the relevant subset using:

\begin{verbatim}
latlim = [lat_min, lat_max];
lonlim = [lon_min, lon_max];
[Z_sub, R_sub] = geocrop(Z, R, latlim, lonlim);
\end{verbatim}

\section{Understanding SRTM Uncertainty Characteristics}
\begin{itemize}
    \item \textbf{Absolute Vertical Accuracy:} Reported as approximately $\pm 16$ m at 90\% confidence.
    \item \textbf{Relative Vertical Accuracy:} Typically between $\pm 6$ m to $\pm 10$ m over shorter distances.
\end{itemize}

\section{Computing Local Uncertainty Metrics}
\subsection{Standard Deviation Method}
Calculate the standard deviation of elevation values as a measure of local variability:

\begin{verbatim}
height_std = std(Z_sub(:), 'omitnan');
\end{verbatim}

\subsection{Slope-Dependent Uncertainty}
Uncertainty often increases with terrain slope. Estimate slope using:

\begin{verbatim}
[slope, aspect] = gradientm(Z_sub, R_sub);
\end{verbatim}

Use an empirical model:

\begin{equation}
\sigma_h = \sigma_0 + k \cdot \tan(\theta)
\end{equation}

Where:
\begin{itemize}
    \item $\sigma_h$ = height uncertainty
    \item $\sigma_0$ = baseline uncertainty (6–10m for SRTM)
    \item $k$ = slope-dependent factor (typically 1–2m per unit slope)
    \item $\theta$ = slope in radians
\end{itemize}

\subsection{Comparison with Ground Truth (if available)}
If ground truth data is available, compute the Root Mean Square Error (RMSE):

\begin{verbatim}
RMSE = sqrt(mean((Z_sub - Z_ground).^2, 'omitnan'));
\end{verbatim}

\section{Summary of Uncertainty}
Report both the standard deviation and RMSE (if applicable). Consider separate metrics for flat and sloped areas if the region has diverse terrain.

\section{Understanding Absolute and Relative Vertical Uncertainty}
\textbf{Absolute vertical uncertainty} refers to the error in elevation measurements relative to an established vertical datum, such as mean sea level. For SRTM, it is reported as $\pm 16$ m at 90\% confidence.

\textbf{Relative vertical uncertainty} refers to the accuracy of elevation differences between nearby points within a localized area, typically around $\pm 6$ m to $\pm 10$ m for SRTM.

\section{Converting Confidence Level to Variance}
To convert $\pm 16$ meters at 90\% confidence level into a variance:

\subsection{1. Interpret the Confidence Level}
The statement $\pm 16$ meters at 90\% confidence level suggests:

\[
\text{P}(-16 \leq X \leq +16) = 0.90
\]

\subsection{2. Find the Corresponding Z-Score}
For a normal distribution, a 90\% confidence interval corresponds to approximately $\pm 1.645$ standard deviations:

\[
\mu \pm 1.645\sigma = \pm 16
\]

\subsection{3. Calculate the Standard Deviation ($\sigma$)}
Solve for $\sigma$:

\[
\sigma = \frac{16}{1.645} \approx 9.73 \text{ m}
\]

\subsection{4. Calculate the Variance ($\sigma^2$)}
Variance is the square of the standard deviation:

\[
\sigma^2 = (9.73)^2 \approx 94.7 \text{ m}^2
\]

\subsection{Summary}
The variance corresponding to $\pm 16$ meters at 90\% confidence is approximately $94.7 \text{ m}^2$.

\end{document}
