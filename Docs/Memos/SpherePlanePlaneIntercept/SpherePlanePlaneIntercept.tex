\documentclass[12pt]{article}
\usepackage{amsmath, amssymb, graphicx, hyperref, tikz, import, cite}
\usepackage[document]{ragged2e}


\title{\textbf{Intersection of Two Planes and a Sphere}}
\author{Douglas O. Carlson, Ph.D.}
\date{\today}

\begin{document}
	\maketitle
	
	\section{Introduction}
	A closed form equation for the intersection of two planes and a sphere is derived. The purpose is to estimate the elevation angle $\varepsilon$ of a target from independent range $r$ and azimuth $\alpha$ measurements in respective East-North-Up (ENU) coordinates. The derivation will take place in global Earth-Centered Earth-Fixed (ECEF) coordinates.
	
	\section{Diagram}
	Here is a circle with two vectors at offset angles and the angle $\alpha$ labeled from points relative to North in an E-N coordinate system.
	
	\begin{center}
		\subimport{./Diagram2}{CircleVecVec.tex}
	\end{center}
	
	\section{General Equations}
	Baseline $\vec{b}^c$ of the two radar platforms in ECEF coordinates; $\vec{P_1}^c$ and $\vec{P_2}^c$.
	\begin{equation}
		\vec{b}^c = \vec{P_2}^c - \vec{P_1}^c
	\end{equation}

	Center of a sphere $\vec{P_0}$.
	\begin{equation}
		\vec{P_0}^c = (\vec{P_1}^c + \vec{P_2}^c)/2
	\end{equation}
	
	Write $\vec{P_1}^c$ and $\vec{P_1}^c$ in terms of $\vec{b}^c$ and $\vec{P_0}^c$.
	\begin{equation}
		\vec{P_1}^c = \vec{P_0}^c - \vec{b}^c / 2
	\end{equation}	
	\begin{equation}
		\vec{P_2}^c = \vec{P_0}^c + \vec{b}^c / 2
	\end{equation}	
	
	Given measurements $r_1$ and $r_2$, write these in terms the unknown point 
	$\vec{P}^c$, and known vectors $\vec{b}^c$ and $\vec{P_0}^c$.
	
	\begin{equation}
		r_1 = |\vec{P}^c - \vec{P_1}^c| = |\vec{P}^c - \vec{P_0}^c + \vec{b}^c/2|
	\end{equation}
	
	\begin{equation}
		r_2 = |\vec{P}^c - \vec{P_2}^c| = |\vec{P}^c - \vec{P_0}^c - \vec{b}^c/2|
	\end{equation}

	Square both sides and substitute $\vec{Q}^c \equiv \vec{P}^c - \vec{P_0}^c$.
	
	\begin{equation}
		r_1^2 = Q^2 + \vec{b}^c \cdot \vec{Q}^c + b^2/4
	\end{equation}

	\begin{equation}
		r_2^2 = Q^2 - \vec{b}^c \cdot \vec{Q}^c + b^2/4
	\end{equation}
	
	Combine the above equations.
	\begin{equation}
		\vec{b}^c \cdot \vec{Q}^c=\frac{r_1^2-r_2^2}{2}
	\end{equation}
	
	\begin{equation}
		r_q^2=|\vec{Q}|^2=\left|\vec{P}-\vec{P}_0\right|^2=\frac{1}{2}\left(r_1^2+r_2^2\right)-\frac{1}{4} b^2
	\end{equation}	
	
	Introduce planes defined by azimuth measurements $\alpha_1$  and $\alpha_2$, and unknown elevation angles
	$\varepsilon_1$  and $\varepsilon_2$. The vectors $\vec{P}^{ENU}$ are defined as target positions in respective ENU
	coordinate systems.
	
	\begin{equation}
		\vec{P}^{ENU} = r\left(\begin{array}{c}
			\sin\alpha \cos\varepsilon \\
			\cos\alpha \cos\varepsilon \\
			\sin\varepsilon
			\end{array}\right)
			\label{eq:pos_vec_enu}
	\end{equation}	
	
	\begin{equation}
		\vec{P}^c = \vec{P_1}^c + [CE_1] \vec{P}^{ENU}(r_1,\alpha_1,\epsilon_1)
	\end{equation}	
	\begin{equation}
		\vec{P}^c = \vec{P_2}^c + [CE_2] \vec{P}^{ENU}(r_2,\alpha_2,\epsilon_2)
	\end{equation}

	The matrices $[EC]$ are defined to transform from global Earth-Centered Earth-Fixed (ECEF) coordinates to local
	ENU. The matrix $[CE]$ is $[EC]^T$
	
	\section{Equation of Intersecting Planes}
	In this section we dispense of vector superscript notation in favor of covariant vector component notation. Superscript e denotes vectors in an ENU coordinate systems and superscript c, those in ECEF.
	\begin{equation}
	P^c = P_m^c + \left[ C E_m \right]_e^c P_m^e  \quad m = 1, 2
	\end{equation}
	
	\begin{equation}
	P_m^e = \left[ E_m C \right]_c^e \left( P^c - P_m^c \right) = \left[ E_m C \right]_c^e \left( Q^c - (-1)^m \frac{b^c}{2} \right)
	\label{eq:pos_vec_enu_2}
	\end{equation}
	
	\begin{equation}
	\vec{P} - \vec{P}_m = \vec{Q} - (-1)^m \frac{\vec{b}}{2} \quad m = 1, 2
	\end{equation}
	
	From Eqn. \ref{eq:pos_vec_enu},
	$
	\cos\alpha_m P_m^{e=1} = \sin\alpha_m P_m^{e=2}.
	$
	and expanding Eqn. \ref{eq:pos_vec_enu_2},
	
	\begin{equation}
		\left( \cos\alpha_m \left[ E_m C \right]_c^1 - \sin\alpha_m \left[ E_m C \right]_c^2 \right) Q^c = 
		 (-1)^{m} \frac{1}{2} \left( \cos\alpha_m \left[ E_m C \right]_c^1 - \sin\alpha_m \left[ E_m C \right]_c^2 \right) b^c
	\end{equation}
	
	\begin{equation}
	a_m \equiv \cos \alpha_m \left[ E_m C \right]_c' - \sin \alpha_m \left[ E_m C \right]_c^2
	\end{equation}
	
	\begin{equation}
	\vec{a}_m \cdot \vec{Q} = (-1)^m \frac{1}{2} \vec{a}_m \cdot \vec{b}
	\end{equation}
	
	\begin{equation}
	\vec{a}_m \cdot \vec{a}_n = 1
	\end{equation}
	
	\begin{equation}
	\vec{Q} \cdot \vec{Q} = r_q^2
	\end{equation}
	

	\section{Appendix A: Algorithm for Plane-Plane-Sphere Intersection \cite{OpenAI2023}}
	
	% Step 1: Intersection Line of Two Planes
	Consider the equations of two planes:
	\begin{align}
		a_1 x + b_1 y + c_1 z &= d_1 \label{eq:plane1} \\
		a_2 x + b_2 y + c_2 z &= d_2 \label{eq:plane2}
	\end{align}
	
	The direction vector of the intersection line is given by the cross product of the normal vectors of the planes:
	
	
	\[
	\mathbf{d} = \begin{vmatrix}
		\mathbf{i} & \mathbf{j} & \mathbf{k} \\
		a_1 & b_1 & c_1 \\
		a_2 & b_2 & c_2
	\end{vmatrix} = (b_1 c_2 - c_1 b_2, c_1 a_2 - a_1 c_2, a_1 b_2 - b_1 a_2)
	\]
	
	
	
	Let \( \mathbf{P}(x_0, y_0, z_0) \) be a point on the intersection line.
	
	% Step 2: Parameterize the Intersection Line
	The intersection line can be parameterized as:
	
	
	\[
	\mathbf{L}(t) = (x_0 + t d_x, y_0 + t d_y, z_0 + t d_z)
	\]
	
	
	
	% Step 3: Substitute into Sphere Equation
	Consider the sphere with equation:
	
	
	\[
	(x - x_s)^2 + (y - y_s)^2 + (z - z_s)^2 = r^2
	\]
	
	
	
	Substitute the parameterized line equation into the sphere equation:
	
	
	\[
	(x_0 + t d_x - x_s)^2 + (y_0 + t d_y - y_s)^2 + (z_0 + t d_z - z_s)^2 = r^2
	\]
	
	
	
	% Step 4: Solve the Quadratic Equation
	Expand and simplify to form a quadratic equation in \( t \):
	
	
	\[
	A t^2 + B t + C = 0
	\]
	
	
	where
	\begin{align*}
		A &= d_x^2 + d_y^2 + d_z^2 \\
		B &= 2 \left( (x_0 - x_s) d_x + (y_0 - y_s) d_y + (z_0 - z_s) d_z \right) \\
		C &= (x_0 - x_s)^2 + (y_0 - y_s)^2 + (z_0 - z_s)^2 - r^2
	\end{align*}
	
	Solve the quadratic equation for \( t \):
	
	
	\[
	t = \frac{-B \pm \sqrt{B^2 - 4AC}}{2A}
	\]
	
	
	
	% Step 5: Find Intersection Points
	The points of intersection \( \mathbf{Q} \) are given by:
	\begin{align*}
		Q_1 &= (x_0 + t_1 d_x, y_0 + t_1 d_y, z_0 + t_1 d_z) \\
		Q_2 &= (x_0 + t_2 d_x, y_0 + t_2 d_y, z_0 + t_2 d_z)
	\end{align*}
	
	\bibliographystyle{ieeetr}
	\bibliography{references}
	
\end{document}	
